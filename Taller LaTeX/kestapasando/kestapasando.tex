\documentclass[11pt,article,oneside]{memoir}

\usepackage[utf8]{inputenc}
\usepackage[spanish,es-nodecimaldot]{babel}

\usepackage{mathtools} % Carga amsmath
\usepackage{amssymb}
\usepackage{amsthm}
\usepackage{bbm}
\usepackage{wasysym}

% Otra fuente
\usepackage[T1]{fontenc}
\usepackage{newtxtext,newtxmath}
% Usamos los paquetitos de la otra persona

\usepackage{hyperref}
\usepackage{enumitem} % IMPORTANTE
\usepackage{microtype}
%\usepackage{showframe}

\hypersetup{pdfborder = {0 0 0}}

% Márgenes
\setulmarginsandblock{2.5cm}{2.5cm}{*}
\setlrmarginsandblock{2.5cm}{2.5cm}{*}
\checkandfixthelayout

\title{Algoritmos de simulación}
\author{inm}
\date{}

\pagestyle{simple}

% Ambientes de teoremas
\theoremstyle{plain}
\newtheorem{theorem}{Teorema}
\newtheorem{corollary}[theorem]{Corolario}
\newtheorem{lemma}[theorem]{Lema}
\newtheorem{proposition}[theorem]{Proposición}
\theoremstyle{definition}
\newtheorem{definition}[theorem]{Definición}
\newtheorem{example}{Ejemplo}
\theoremstyle{remark}
\newtheorem{remark}[theorem]{Observación}

% Declaración de comandos y operadores
\newcommand\RR{\mathbb R}
\newcommand\1{\mathbbm 1}
\newcommand\PP{\mathbb P}
\newcommand\dd{\,\mathrm d}
\DeclareMathOperator\cov{Cov}
\DeclareMathOperator\var{Var}
\DeclareMathOperator\gauss{N}
\DeclareMathOperator\unif{U}
\DeclareMathOperator\expo{Exp}
\DeclareMathOperator\gam{Ga}
\DeclareMathOperator\cau{Cauchy}
\DeclareMathOperator\dbeta{Beta}

% Operadores "especiales"
\newcommand\abs[1]{\ensuremath{\left\lvert #1 \right\rvert}}

\renewcommand{\qedsymbol}{$\frownie$}

\begin{document}
\maketitle

\noindent Algunos recordatorios de las relaciones entre variables aleatorias.

\begin{lemma}
    Las siguientes relaciones son verdaderas.
    \begin{enumerate}
        \item Si $U \sim \mathrm \unif(0,1)$ y $\lambda > 0$ entonces $-\log(U)/\lambda \sim \expo(\lambda)$.
        \item Si $Z \sim \gauss(0,1)$ entonces $Z^2 \sim \chi_{1}^2 = \gam(1/2, 1/2)$.
        \item Si $Z_1, Z_2 \sim \gauss(0,1)$ son independientes entonces $Z_2 / Z_2 \sim \cau(0,1)$.
    \end{enumerate}
\end{lemma}

\begin{proof}
    Para establecer el inciso 1, consideremos $U \sim \unif(0,1)$, denotemos $X = -\log(U)/\lambda$ y observemos que para $x < 0$, $\PP(X \leq x) = 0$, mientras que para $x \geq 0$, 
    \[
        \PP(X \leq x) = \PP(\log(U) \geq -\lambda x) = \PP(U \geq e^{-\lambda x}) = 1 - e^{-\lambda x}.
    \]

    Para ver que el inciso 2 es verdadero supongamos que $Z \sim \gauss(0,1)$. Para $x < 0$, $\PP(Z^2 \leq x) = 0$; por otra parte, si $x \geq 0$ entonces
    \[
        \PP(Z^2 \leq x) = \PP(-\sqrt x \leq Z \leq \sqrt x) = \Phi(\sqrt x) - \Phi(-\sqrt x),
    \]
    donde la última igualdad se da por la continuidad de $Z$ y la definición de $\Phi$. Así pues, cuando $x \geq 0$, derivando con respecto a $x$ y usando la densidad de una normal estándar obtenemos que 
    \[
        f_{Z^2}(x) = \frac{1}{\sqrt{2\pi x}} e^{-\frac 1 2 x} = \frac{\left(\frac 1 2 \right)^\frac{1}{2}}{\Gamma \left(\frac 1 2\right)} x^{\frac{1}{2} - 1} e^{-\frac 1 2 x},
    \]
    donde hemos usado que $\Gamma(1/2) = \sqrt \pi$.

    Para concluir, recordemos que si $Y$ y $W$ son variables aleatorias continuas e independientes, entonces 
    \[
        f_{W/Y}(x) = \int_{-\infty}^\infty \abs{y} f_Y(y) f_W(wy) \dd y.
    \]
    En efecto, para $x \in \RR$ tendremos que
    \begin{align*}
        \PP\left(\frac W Y \leq x \right) & = \iint_{\{w/y \leq x\}} f_Y(y) f_W(w) \dd w\dd y \\
        & = \int_{-\infty}^0 \int_{xy}^\infty f_Y(y) f_W(w) \dd w\dd y + \int_0^\infty \int_{-\infty}^{xy} f_Y(y) f_W(w) \dd w\dd y \\
        & = \int_{-\infty}^0 \int_x^{-\infty} y f_Y(y) f_W(zy) \dd z\dd y + \int_0^\infty \int_{-\infty}^x y f_Y(y) f_W(zy) \dd z\dd y \\
        & = \int_{-\infty}^x \left( \int_{-\infty}^0 (-y) f_Y(y) f_W(zy) \dd y +  \int_0^\infty y f_Y(y) f_W(zy) \dd y \right) \dd z \\
        & = \int_{-\infty}^x \int_{-\infty}^\infty \abs{y} f_Y(y) f_W{zy} \dd y\dd z,
    \end{align*}
    donde la tercera igualdad se da por el cambio de variable $z = w/y$ y la cuarta igualdad es Fubini y un poco de álgebra. Así las cosas, si $C = Z_2 / Z_1$, entonces para $x \in \RR$, 
    \[
        f_C(x) = \int_{-\infty}^\infty \abs{y} \frac{1}{\sqrt{2\pi}} e^{-\frac{y^2}{2}} \frac{1}{\sqrt{2\pi}} e^{-\frac{(xy)^2}{2}} \dd y = \frac 1 \pi \int_0^\infty y e^{- \frac{(1+x^2)y^2}{2}} \dd y = \frac{1}{\pi(1 + x^2)},
    \]
    que es la densidad de una variable aleatoria $\cau(0,1)$.
\end{proof}

% Podemos importar textos .tex directamente sin copiar y pegar :O
% La primera manera: input
En un curso de probabilidad suele verse el método de Box--Muller, el cual nos dice que si $U_1, U_2 \sim \unif(0,1)$ son independientes entonces la transformación 
\begin{align*}
    Z_1 & = \sqrt{-2\log(U_1)} \cos(2\pi U_2), \\
    Z_2 & = \sqrt{-2\log(U_1)} \sen(2\pi U_2),
\end{align*}
nos da un vector aleatorio $(Z_1, Z_2)$ con $Z_1, Z_2 \sim \gauss(0,1)$ independientes. Analicemos estas expresiones detenidamente y observemos que 
\begin{align*}
    -2\log(U_1) & = Z_1^2 + Z_2^2, \\
    \tan(2 \pi U_2) & = Z_2 /Z_1.
\end{align*}
Del recordatorio sabemos que $Z_i^2 \sim \gam(1/2, 1/2)$, por lo que $Z_1^2+Z_2^2 \sim \gam(1, 1/2) = \exp(1/2)$, lo cual concuerda con que $-2\log(U_1) \sim \expo(1/2)$. ¡Pero tenemos más! Observemos que $Z_2/Z_1 \sim \cau(0,1)$, por lo que $\tan(2\pi U_2) \sim \cau(0,1)$, es decir que hemos encontrado una forma de simular variables aleatorias Cauchy. 

Ahora, podemos aprovechar el método de Box--Muller para obtener variables normales independientes no estándar sin morir en el intento. Pues si $\lambda_1, \lambda_2 > 0$ y $U_1, U_2 \sim \unif(0,1)$ son independientes, entonces el vector aleatorio $(X_1, X_2)$, definido por la transformación
\begin{align*}
    X_1 & = \sqrt{-\frac 1 {\lambda_1} \log (U_1)} \cos(2\pi U_2), \\
    X_2 & = \sqrt{-\frac 1 {\lambda_2} \log (U_1)} \sen(2\pi U_2),
\end{align*}
es tal que $X_1$ y $X_2$ son variables aleatorias independientes tales que $X_i \sim \gauss(0, (2\lambda_i)^{-1})$, puesto que se dará la igualdad $X_i = (2\lambda_i)^{-1/2} Z_i$.

Dejando de lado Box--Muller, algunas otras consecuencias del recordatorio, considerando una sucesión $\{U_i\}_{i \geq 1}$ de variables aleatorias independientes e idénticamente distribuidas $\unif(0,1)$, son las siguientes:
\begin{enumerate}
    \item Si $\nu \geq 1$ es entero, $Y = - 2 \sum_{j = 1}^\nu \log(U_j) \sim \chi_{2\nu}^2 = \gam((2\nu)/2, 1/2)$.
    \item Si $\alpha \geq 1$ es entero y $\lambda > 0$, $Y = -\sum_{j = 1}^\alpha \log(U_j) / \lambda \sim \gam(\alpha, \lambda)$.
    \item Si $\alpha, \beta \geq 1$ son enteras, entonces 
    \[
        Y = \frac{\sum_{j = 1}^\alpha \log(U_j)}{\sum_{j = 1}^{\alpha+\beta} \log(U_j)} \sim \dbeta(\alpha, \beta).
    \]
\end{enumerate}
Los primeros dos resultado son claros y se dejan al lector. Para ver que el tercer resultado es cierto, probaremos algo más general: para $\alpha, \beta > 0$, si $Y_1 \sim \gam(\alpha, 1)$ y $Y_2 \sim \gam(\beta, 1)$ son independientes, entonces $Y_1/(Y_1+Y_2) \sim \dbeta(\alpha, \beta)$ será independiente de $Y_1 + Y_2 \sim \gam(\alpha + \beta, 1)$. Con este fin, definamos $W = Y_1/(Y_1 + Y_2)$ y $S = Y_1 + Y_2$. Del teorema de cambio de variable, 
\begin{align*}
    f_{W, S}(w,s) & = s f_{Y_1}(ws) f_{Y_2}(s-ws) \\
    & = s \times \frac{1}{\Gamma(\alpha)} (ws)^{\alpha - 1} e^{-ws} \1_{\RR_+}(ws) \times \frac{1}{\Gamma(\beta)} (s-ws)^{\beta - 1} e^{s-ws} \1_{\RR_+}(s-ws) \\
    & = \left( \frac{\Gamma(\alpha+\beta)}{\Gamma(\alpha) \Gamma(\beta)} w^{\alpha-1} (1-w)^{\beta - 1} \1_{(0,1)}(w) \right) \times \left( \frac{1}{\Gamma(\alpha + \beta)} s^{(\alpha + \beta) - 1} e^{-s} \1_{\RR_+}(s) \right).
\end{align*}
Que el inciso 3 es cierto es ahora inmediato.
% La segunda: include
Para finalizar veremos un algoritmo en sumo útil para generar variables aleatorias continuas con solo conocer un múltiplo de la función de densidad: el \emph{método de aceptación y rechazo}. Supongamos que $f$ es una función de densidad y que $g$ es otra función de densidad tal que $f \leq M g$ para alguna $M > 0$. El algoritmo consiste en los siguientes pasos:
\begin{enumerate}
    \item Generamos una variable aleatoria $X$ de la función de densidad $g$ y una variable aleatoria $U \sim \unif(0,1)$.
    \item Si $U \leq f(X) / (M g(X))$, entonces definimos $Y = X$ y terminamos.
    \item En otro regresamos a 1.
\end{enumerate}

\begin{proposition} \label{prop:aceptacion y rechazo}
    La variable aleatoria $Y$ generada mediante el algoritmo de aceptación y rechazo tiene densidad $f$. Más aún, la cantidad de simulaciones que haremos para aceptar un valor $X$ es una variable aleatoria geométrica con parámetro $1/M$.
\end{proposition}

\begin{proof}
    Obtengamos la función de distribución de $Y$. 
    \begin{align*}
        \PP(Y \leq y) & = \PP \left( X \leq y \,\Bigg\vert\, U \leq \frac{f(X)}{M g(X)} \right) 
        = \frac{\PP \left( X \leq y, U \leq \frac{f(X)}{Mg(X)} \right)}{\PP \left(U \leq \frac{f(X)}{Mg(X)} \right)}.
    \end{align*}
    Para el numerador tendremos, por independencia,
    \begin{align*}
        \PP \left( X \leq y, U \leq \frac{f(X)}{Mg(X)} \right) & = \int_{-\infty}^y \PP\left(U \leq \frac{f(x)}{Mg(x)}\right) g(x) \dd x = \int_{-\infty}^y g(x) \int_0^{f(x)/Mg(x)} \dd u \dd x \\
        & = \int_{-\infty}^y g(x) \frac{f(x)}{Mg(x)} \dd x = \frac 1 M \int_{-\infty}^y f(x) \dd x,
    \end{align*}
    mientras que el denominador estará dado por 
    \begin{align}
        \PP \left(U \leq \frac{f(X)}{Mg(X)} \right) & = \lim_{y \to \infty} \frac 1 M \int_{-\infty}^y f(x) \dd x = \frac 1 M \int_{-\infty}^\infty f(x) \dd x = \frac 1 M. \label{eq:geometrica}
    \end{align}
    Así las cosas, $\PP(Y \leq y) = \int_{-\infty}^y f(x) \dd x$ y por lo tanto $f_Y(y) = f(y)$. Más aún, de (\ref{eq:geometrica}) vemos que la probabilidad de aceptar una simulación $X$ como simulación de $Y$ es $1/M$, de donde se sigue fácilmente el resultado.
\end{proof}

Una consecuencia importante de la demostración de la proposición \ref{prop:aceptacion y rechazo} es que en realidad no es necesario conocer $f$ en su totalidad, basta conocer un múltiplo de la función de densidad. Esto es útil en estadística bayesiana, pues para un valor de $x$ fijo se tiene la relación de proporcionalidad
\[
    f_{\Theta \,\vert\, X}(\theta \,\vert\, x) \propto f_{X \,\vert\, \Theta}(x \,\vert\, \theta) f_\Theta(\theta) = f_{X, \Theta}(x, \theta),
\]
por lo cual para realizar simulaciones de la \emph{distribución posterior} de $\Theta$, basta conocer la densidad conjunta y no hará falta normalizar. El costo a pagar será encontrar una función de densidad $g$ y una $M > 0$ tales que $f_{X, \Theta}(x, \theta) \leq M g(\theta)$ para todo $\theta$.

\begin{example}[Simulación de gaussianas estándar de una Cauchy]
    Cuando $f(x) = (2\pi)^{-1/2} \exp(-x^2/2)$ y $g(x) = (\pi(1+x^2))^{-1}$,
    \[
        \frac{f(x)}{g(x)} = \sqrt\frac{\pi}{2} (1+x^2) e^{-\frac{x^2}{2}}
    \]
    está acotado por $M = \sqrt{2\pi/e}$, valor que se alcanza en $x = \pm 1$, por lo que la probabilidad de aceptación es $1/M = \sqrt{e/2\pi} \approx 0.66$, lo cual implica que en promedio, una de cada tres simulaciones de la Cauchy es rechazada.
\end{example}

\begin{example}[Generación de variables aleatorias gamma]
    Supongamos que $\alpha \geq 1$ y, sin pérdida de generalidad, supongamos que $\lambda = 1$. Entonces $f(x) = x^{\alpha - 1} e^{-x} / \Gamma(\alpha)$. Proponemos $g_b(x) = b^a x^{a-1} e^{-bx} / \Gamma(a)$, con $a = \lfloor \alpha \rfloor$ y $b < 1$. Entonces 
    \[
        \frac{f(x)}{g_b(x)} = \frac{\Gamma(a)}{\Gamma(\alpha)} b^{-a} x^{\alpha - a} e^{-(1-b)x} \leq \frac{\Gamma(a)}{\Gamma(\alpha)} b^{-a} \left( \frac{\alpha-a}{(1-b)e} \right)^{\alpha-a},
    \]
    y vemos que la cota mínima se alcanza con $b = a / \alpha$.
\end{example}

\end{document}