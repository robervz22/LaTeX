\chapter{Electric Boogaloo}

\noindent Después de esto viene la bibliografía.

\section{Matemáticas}

\noindent Algunos ambientes matemáticos se muestran a continuación.

\begin{defn}
    Soy la definición necesaria.
\end{defn}

\begin{thm}
    Soy el teorema importante.
\end{thm}

Como todos saben, un teorema debe ser demostrado. A continuación se da la prueba:

\Proof
    Se deja como ejercicio al lector. \qed


Si la prueba termina en una ecuación hay que usar \texttt{\textbackslash{}qed} para que el símbolo de que terminó la prueba salga en la misma línea que la ecuación:

\Proof
    Evidentemente se cumple que 
    \[
        a^3 + b^3 = c^3. \qed
    \]
