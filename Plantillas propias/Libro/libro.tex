\documentclass[
    11pt, 
    leqno, 
    twoside, 
    b5paper,
    mpinclude=true,
    abstract=true
    ]{scrreport}

%%%%%%%%%%%%%%%%%%%%%%%%%%%%%%%%%%%%%%%%%%%%%%%%%%%%%%%%%%%%%%%%%%%%%%%%%%%%%%%%%
% Opciones de inputs (útil si no se compila con LuaLaTeX o XeLaTeX)
%\usepackage[utf8]{inputenc}

%%%%%%%%%%%%%%%%%%%%%%%%%%%%%%%%%%%%%%%%%%%%%%%%%%%%%%%%%%%%%%%%%%%%%%%%%%%%%%%%%
% Idioma del documento
\usepackage[spanish,es-nodecimaldot,es-tabla,es-lcroman]{babel}

%%%%%%%%%%%%%%%%%%%%%%%%%%%%%%%%%%%%%%%%%%%%%%%%%%%%%%%%%%%%%%%%%%%%%%%%%%%%%%%%%
% Permitir más alfabetos matemáticos
\newcommand\hmmax{0}
\newcommand\bmmax{0}

%%%%%%%%%%%%%%%%%%%%%%%%%%%%%%%%%%%%%%%%%%%%%%%%%%%%%%%%%%%%%%%%%%%%%%%%%%%%%%%%%
% Carga el archivo macros.sty que tiene definiciones que se usan más adelante
\usepackage{macros}

%%%%%%%%%%%%%%%%%%%%%%%%%%%%%%%%%%%%%%%%%%%%%%%%%%%%%%%%%%%%%%%%%%%%%%%%%%%%%%%%%
% Opciones de la fuente pdfLaTeX
%\usepackage{fouriernc}
%\usepackage[scale=0.9]{plex-serif}
%\usepackage[scaled=0.85]{FiraSans}

%%%%%%%%%%%%%%%%%%%%%%%%%%%%%%%%%%%%%%%%%%%%%%%%%%%%%%%%%%%%%%%%%%%%%%%%%%%%%%%%%
% Fuente con LuaLaTex o XeLaTeX, preferentemente LuaLaTeX por microtype
\usepackage{fontspec}
\usepackage{fourier-otf}
\usepackage[
    DefaultFeatures={Scale=MatchLowercase},
    RMstyle={Text,Semibold},
    SSstyle={Text,Semibold},
    SSconstyle={Text,Semibold},
    TTstyle={Text,Semibold}]{plex-otf}
\setsansfont{FiraSans}[Scale=MatchLowercase]
\setmathfont{Erewhon Math}[Style=leqslant,Scale=MatchUppercase]

%%%%%%%%%%%%%%%%%%%%%%%%%%%%%%%%%%%%%%%%%%%%%%%%%%%%%%%%%%%%%%%%%%%%%%%%%%%%%%%%%
% Datos del trabajo
\newcommand{\mititulo}{Título del trabajo}
\newcommand{\materia}{Mi materia favorita}
\newcommand{\elautor}{Mi nombre, supongo}
\author{\elautor}
\title{\trabajo}
\date{}

%%%%%%%%%%%%%%%%%%%%%%%%%%%%%%%%%%%%%%%%%%%%%%%%%%%%%%%%%%%%%%%%%%%%%%%%%%%%%%%%%
% Paquetería para floats: 
\usepackage{graphicx}
\usepackage{booktabs}
% Quitar el número de la sección de los contadores de figuras y tablas
\counterwithout{figure}{section}
\counterwithout{table}{section}
% Fuente de las palabras Figura y Tabla
\setkomafont{captionlabel}{\sffamily}

%%%%%%%%%%%%%%%%%%%%%%%%%%%%%%%%%%%%%%%%%%%%%%%%%%%%%%%%%%%%%%%%%%%%%%%%%%%%%%%%%
% Márgenes
\usepackage[margin=2cm]{geometry}

%%%%%%%%%%%%%%%%%%%%%%%%%%%%%%%%%%%%%%%%%%%%%%%%%%%%%%%%%%%%%%%%%%%%%%%%%%%%%%%%%
% Formato de la encabezados y pies de página
\usepackage[automark]{scrlayer-scrpage}% sets pagestyle scrheadings automatically
\clearpairofpagestyles
\ofoot*{\pagemark}
\rehead*[]{\mititulo}
\lehead*[]{\headmark}
\lohead*[]{\materia}
\rohead*[]{\headmark}

% Pie de página con líneas para separar el encabezado y el pie de página
% del resto
\KOMAoptions{
  footsepline=1pt:.25\paperwidth,% syntax: footsepline=<thickness>:<length>
  plainfootsepline,
  olines,
  headsepline=0.5pt:\textwidth
}
% Recorre la línea a la parte exterior de la hoja
\ModifyLayer[
  hoffset=0pt,
  width=\paperwidth,
  addvoffset=-5pt% move the line up
]{scrheadings.foot.above.line}
\RedeclareLayer[
  clone=scrheadings.foot.above.line
]{plain.scrheadings.foot.above.line}

% Fuente de pies de página y encabezados
\setkomafont{pagenumber}{\normalfont\sffamily\bfseries}
\setkomafont{pagehead}{\sffamily}
%\renewcommand*{\sectionmarkformat}{\S\thesection\autodot}
\automark*[section]{chapter}

%%%%%%%%%%%%%%%%%%%%%%%%%%%%%%%%%%%%%%%%%%%%%%%%%%%%%%%%%%%%%%%%%%%%%%%%%%%%%%%%%
% Paquetería adicional
\usepackage{microtype}
\usepackage{multicol}
\usepackage{anyfontsize}
\usepackage{ragged2e}
\usepackage{comment}
\usepackage{pdfpages}
\usepackage{lipsum}

%%%%%%%%%%%%%%%%%%%%%%%%%%%%%%%%%%%%%%%%%%%%%%%%%%%%%%%%%%%%%%%%%%%%%%%%%%%%%%%%%
% Opciones de enlistados
\usepackage[inline]{enumitem}
% Los enlistados ahora tienen (i), (ii), ... en el primer nivel y (a), (b), ...
% en el segundo nivel
\setlist[1]{parsep=0pt,
topsep=3pt,
itemsep=0pt,
label=\textup{(\roman*)},
leftmargin=*}
\setlist[2]{parsep=0pt,
topsep=3pt,
itemsep=0pt,
label=\textup{(\alph*)}, 
leftmargin=*}

%%%%%%%%%%%%%%%%%%%%%%%%%%%%%%%%%%%%%%%%%%%%%%%%%%%%%%%%%%%%%%%%%%%%%%%%%%%%%%%%%
% Redefinir el ambiente de demostración para que la fuente de la palabra 
% Demostración sea sans

\usepackage{etoolbox}
\makeatletter

\renewenvironment{proof}[1][\proofname]{\par
  \pushQED{\qed}%
  \normalfont \topsep6\p@\@plus6\p@\relax
  \trivlist\item\relax{
    \sffamily#1\@addpunct{.}}\hspace\labelsep\ignorespaces
  }{%
  \popQED\endtrivlist\@endpefalse}
\makeatother
% Redefinir el símbolo al final de una demostración
\renewcommand{\qedsymbol}{$\blacksquare$}

%%%%%%%%%%%%%%%%%%%%%%%%%%%%%%%%%%%%%%%%%%%%%%%%%%%%%%%%%%%%%%%%%%%%%%%%%%%%%%%%%
% TikZ y PGFPLots
\usepackage{tikz}
\usetikzlibrary{
    knots,
    hobby,
    decorations.pathreplacing,
    shapes.geometric,
    calc,
    shapes
    }
\usepackage{pgfplots}
\pgfplotsset{compat=1.7}
\usepackage{pgfornament}

%%%%%%%%%%%%%%%%%%%%%%%%%%%%%%%%%%%%%%%%%%%%%%%%%%%%%%%%%%%%%%%%%%%%%%%%%%%%%%%%%
% Bibliografía con BibLaTeX
\usepackage[backend=biber,style=apa]{biblatex}
\usepackage{csquotes}
% Usamos el formato APA
\DeclareLanguageMapping{spanish}{spanish-apa}
\urlstyle{same}
% Cargamos la fuente con los datos bibliográficos
\addbibresource{demo.bib}

%%%%%%%%%%%%%%%%%%%%%%%%%%%%%%%%%%%%%%%%%%%%%%%%%%%%%%%%%%%%%%%%%%%%%%%%%%%%%%%%%
% Estilo de los capítulos
% Definición de colores
\definecolor{mytitlecolor}{RGB}{115,35,60}
\definecolor{mysectioncolor}{RGB}{104,34,139}
% Cambiar la fuente del capítulo
\setkomafont{chapter}{\sffamily\huge\bfseries}
\renewcommand*{\raggedchapter}{\raggedleft}
% Definir el estilo del capítulo tipo Bringhurst
\makeatletter
\newsavebox{\feline@chapter}
\newcommand\feline@chapter@marker[1][4cm]{%
  \sbox\feline@chapter{%
    \resizebox{!}{#1}{\setlength{\fboxsep}{0pt}%
        \colorbox{white}{\color{mytitlecolor}\thechapter}}}%
  \raisebox{-0.9cm}{\usebox{\feline@chapter}}%
}
\renewcommand*{\chapterformat}{\feline@chapter@marker[1.6cm]}

\renewcommand\chapterlinesformat[3]{%
  \Ifstr{#1}{chapter}
  {%
    \makebox[\textwidth][l]{%
      \parbox[b]{\textwidth}{\raggedchapter\color{mytitlecolor} #3}%
      \hspace*{\marginparsep}#2%
    }\\*[-.5\baselineskip]
    \rule{\textwidth}{.4pt}%
    \\[-0.7\baselineskip]
    \par%
  }
  {\@hangfrom{#2}{#3}}%
}
\makeatother
% Redefinir un poco los epígrafes
\renewcommand{\dictumwidth}{0.4\textwidth}
\renewcommand*\dictumauthorformat[1]{(#1)\bigskip}
% Cambiar el color de las secciones
\addtokomafont{section}{\color{mysectioncolor}}
% Que las secciones, subsecciones, etc. también salgan en sans en el índice
\newcommand*\tocentryformat[1]{{\sffamily#1}}
\RedeclareSectionCommands
  [
    tocentryformat=\tocentryformat,
    tocpagenumberformat=\tocentryformat
  ]
  {section,subsection,subsubsection,paragraph,subparagraph}

%%%%%%%%%%%%%%%%%%%%%%%%%%%%%%%%%%%%%%%%%%%%%%%%%%%%%%%%%%%%%%%%%%%%%%%%%%%%%%%%%
% Hipervínculos y enlaces dentro del documento
\usepackage{hyperref}
\hypersetup{
        colorlinks=true,
        linkcolor=mysectioncolor,
        filecolor=magenta,      
        urlcolor=cyan,
        pdfauthor={el autor},
    citecolor=mysectioncolor
}
\usepackage[spanish]{cleveref}

\begin{document}

\begin{titlepage}
    \begin{addmargin}[-1.5cm]{-1.5cm}
        \centering % centrada
        %\raggedleft % Alineada a la derecha
        %\raggedright % Alineada a la izquierda
        
        %\vspace*{\baselineskip} % Línea vertical 
        %\begin{picture}(0,0)
        %  \put(-7cm,-1.3\textheight){\color{coolred}\rule{0.5cm}{1.2\paperheight}}
        %\end{picture}

        \hfill\includegraphics[width=4cm]{example-image-a}
        \vspace*{0.05\textheight} % Whitespace before the title
        
        %------------------------------------------------
        %	Cosas del título
        %------------------------------------------------
            
        \color{coolred}
        \rule{0.8\textwidth}{2pt}
        \vspace*{\baselineskip}

            % Nombre de la materia  
        \textsf{\textbf{\huge \color{mytitlecolor}{\materia}}}  
        \vspace*{0.05\textheight}

        % Aquí va el título
        {\Huge{\textsf{\textbf{\color{mytitlecolor}{\trabajo}}}}}
        \vspace*{\baselineskip}

        \rule{0.8\textwidth}{2pt}

        \vspace*{0.03\textheight}

        %------------------------------------------------
        %	Aquí van los nombres
        %------------------------------------------------
        
        \color{black}
        {\large \textsf{Alumnos:}}  \vspace*{1.5\baselineskip}

        {\Large Alumno \# 1} \vspace*{\baselineskip}

        {\Large Alumno \# 2} \vspace*{\baselineskip}

        {\Large Alumno \# 3} \vspace*{2\baselineskip}

        {\large \textsf{Profesor:}} \vspace*{1.5\baselineskip}

        {\Large Nombre del profesor} \vspace*{2\baselineskip}

        {\large \textsf{Escuela, Ciudad}} \vspace*{\baselineskip}

        {\large \textsf{\fechaentrega}}
        
        \vspace*{\fill}
    \end{addmargin}
\end{titlepage}

\tableofcontents

\chapter{Soy un capítulo}

\dictum[Jorge Cham, \url{www.phdcomics.com}]{Undergradese, I: ``Is it going to be an open book exam?''\\ Translation: ``I don’t have to actually memorize anything, do I?''}

\noindent \lipsum[1-2]

\section{Aquí va un pensamiento}

\noindent \lipsum

\section{Otra sección}

\noindent \lipsum
\chapter{Lo que sea}

\dictum[Sir Terry Pratchett, The Colour of Magic]{Tourist, Rincewind decided, meant idiot.}

\noindent \lipsum[2]

\section{Algo}

\noindent \lipsum
\addchap{A ver qué pasa}

\dictum[Emily St. John Mandel, Station Eleven]{They spend all their lives waiting for their lives to begin.}

\noindent \lipsum[1-2]

\addsec{Lo que sea que vaya aquí}

\noindent \lipsum[1-3]

\appendix
\chapter{Unga bunga}

\noindent Esto es un apéndice.

\section{Una sección en el apéndice}

\noindent \lipsum
\chapter{Electric Boogaloo}

\noindent Después de esto viene la bibliografía.

\section{Matemáticas}

\noindent Algunos ambientes matemáticos se muestran a continuación.

\begin{definition}
    Soy la definición necesaria.
\end{definition}

\begin{theorem}
    Soy el teorema importante.
\end{theorem}

Como todos saben, un teorema debe ser demostrado. A continuación se da la prueba:

\begin{proof}
    Se deja como ejercicio al lector.
\end{proof}

Si la prueba termina en una ecuación hay que usar \texttt{\textbackslash{}qedhere} para que el símbolo de que terminó la prueba salga en la misma línea que la ecuación:

\begin{proof}
    Evidentemente se cumple que 
    \[
        a^3 + b^3 = c^3. \qedhere
    \]
\end{proof}

\nocite{*}
\printbibliography

\end{document}