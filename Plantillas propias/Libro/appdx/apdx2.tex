\chapter{Electric Boogaloo}

\noindent Después de esto viene la bibliografía.

\section{Matemáticas}

\noindent Algunos ambientes matemáticos se muestran a continuación.

\begin{definition}
    Soy la definición necesaria.
\end{definition}

\begin{theorem}
    Soy el teorema importante.
\end{theorem}

Como todos saben, un teorema debe ser demostrado. A continuación se da la prueba:

\begin{proof}
    Se deja como ejercicio al lector.
\end{proof}

Si la prueba termina en una ecuación hay que usar \texttt{\textbackslash{}qedhere} para que el símbolo de que terminó la prueba salga en la misma línea que la ecuación:

\begin{proof}
    Evidentemente se cumple que 
    \[
        a^3 + b^3 = c^3. \qedhere
    \]
\end{proof}